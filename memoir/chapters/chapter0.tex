\chapter*{Introducción}
\addcontentsline{toc}{section}{\textbf{Introducción}}
\markboth{Introducción}{Introducción}
En Topología, una curva de Jordan es una curva cerrada en el plano sin autointersecciones. En definitiva, una copia topológica de la circunferencia en el plano euclidiano. El teorema clásico de la curva de Jordan  afirma que el complemento de una curva de Jordan en el plano tiene dos componentes conexas, una acotada (interior de la curva) y otra no acotada   (exterior de la curva), siendo la curva la frontera topológica común  de ambas componentes. Este resultado resulta intuitivamente obvio para cualquier observador, aunque tras una breve reflexión sobre su naturaleza se comprende  que su prueba dista mucho de ser trivial. El primer matemático que matizó la dificultad implícita en este enunciado fue B. Bolzano. Haciendo un recorrido histórico, la primera demostración rigurosa del teorema de la curva de Jordan conocida se centró en curvas poligonales, luego se generalizó al ambiente de curvas diferenciables, y finalmente   Camille Jordan (1887) la extendió a curvas continuas generales. La prueba original de Jordan no estuvo ajena a la polémica ya que contenía algunas lagunas técnicas. Fue Schönflies (1924) quien  revisó este resultado clásico,  obteniendo una generalización fuerte conocida en la literatura como teorema de Jordan-Schönflies. Este resultado afirma que cualquier homeomorfismo entre   curvas de Jordan (siempre existen tales homeomorfismos ya que   éstas son copias topológicas de la circunferencia) puede ser extendido a un homeomorfismo global del plano en sí mismo. En otras palabras, Schönflies demostró que el embebimiento natural de la circunferencia en al plano es topológicamente rígido. 

El teorema de Jordan-Schönflies es extraordinariamente útil desde el punto de vista de la Topología bidimensional, no siendo un atrevimiento  afirmar que es el pilar sobre el que se sustenta la misma. 
Para comprenderlo recordemos que la Topología en dos dimensiones se dedica fundamentalmente al estudio de las superficies topológicas, esto es, de los espacios topológicos  Hausdorff y localmente euclidianos.  Una de las implicaciones cruciales del teorema de Jordan-Schönflies  ha sido el facilitar la comprensión de la naturaleza celular de las tales objetos, punto clave para su clasificación. Fue Radó quien en 1925  analizó la estructura topológica básica de las superficies, probando que  éstas admiten una descomposición celular elemental, o equivalentemente, que pueden ser  trianguladas. La relevancia del teorema de Radó  reside en que  es la clave para introducir la representación poligonal de las superficies compactas, y por tanto, para  su clasificación completa mediante cirugía topológica.

El objetivo  de este Trabajo Fin de Grado (TFG) ha sido la demostración del teorema de Radó   utilizando el teorema de Jordan-Schönflies como herramienta básica. La memoria que presentamos comprende una ampliación del conocimiento en relación a la asignatura la Topología II del Grado en Matemáticas,   aunque también ha requerido  de algún material complementario de  Álgebra II y Topología I.  

El objeto de mejorar la exposición y hacerla más autocontenida, hemos creído necesario hacer una pequeña introducción incluyendo  una prueba esquemática   del teorema de la curva de Jordan. Este lenguaje nos ha permitido  presentar el teorema de Jordan-Schönflies, junto con algunas de sus generalizaciones básicas, en el contexto más apropiado para su utilización   en la demostración del teorema de Radó. La referencia bibliográfica básica para la realización de este trabajo ha sido el artículo de C. Thomassen, \cite{Thomassen}  mientras que para cuestiones relativas a la Topología general hemos consultado los libros de W. S. Massey \cite{Massey} y W. T. Tutte. \cite{Tutte}
