\documentclass[12pt,a4paper,oldfontcommands]{memoir}
\usepackage[utf8]{inputenc}
\usepackage[spanish]{babel}

\usepackage[T1]{fontenc}
\usepackage{microtype}
\usepackage{graphicx}
\usepackage{wrapfig}
\usepackage{xcolor}
\usepackage{times}
\usepackage{amsmath}
\usepackage{amsfonts}
\usepackage{amssymb}
\usepackage{amsthm}
\usepackage{subfig}
\usepackage[pages=some]{background}
\usepackage{setspace}

\graphicspath{ {Figures/} }

\usepackage[breaklinks=true,colorlinks=true,
linkcolor=black,urlcolor=black,citecolor=black,
bookmarks=true,bookmarksopenlevel=2]{hyperref}

\usepackage{tikz}

\usepackage{geometry}
\geometry{total={210mm,297mm}, left=25mm,right=25mm,bindingoffset=10mm, top=25mm,bottom=25mm}

\OnehalfSpacing

%%% CHAPTER'S STYLE
%\chapterstyle{bianchi}
\chapterstyle{ger}
%\chapterstyle{madsen}
%\chapterstyle{ell}

%%% STYLE OF SECTIONS, SUBSECTIONS, AND SUBSUBSECTIONS
\setsecheadstyle{\Large\bfseries\sffamily\raggedright}
\setsubsecheadstyle{\large\bfseries\sffamily\raggedright}
\setsubsubsecheadstyle{\bfseries\sffamily\raggedright}


%%% STYLE OF PAGES NUMBERING
\pagestyle{companion}\nouppercaseheads 
%\pagestyle{headings}
%\pagestyle{Ruled}
%\pagestyle{plain}
\makepagestyle{plain}
\makeevenfoot{plain}{\thepage}{}{}
\makeoddfoot{plain}{}{}{\thepage}
\makeevenhead{plain}{}{}{}
\makeoddhead{plain}{}{}{}


\maxsecnumdepth{subsection} % chapters, sections, and subsections are numbered
\maxtocdepth{subsection} % chapters, sections, and subsections are in the Table of Contents

\renewcommand{\qedsymbol}{$\blacksquare$}

\backgroundsetup{
scale=1, %escala de la imagen, es recomendable que sea del mismo tamaño que el pdf
color=black, %fondo a usar para transparencia
opacity=0.1, %nivel de transparencia
angle=0, %en caso de querer una rotación
contents={%
	\includegraphics[width=\paperwidth,height=\paperheight]{LogoUGR.jpg} %nombre de la imagen a utilizar como fondo
	}
}

\renewcommand{\baselinestretch}{1.5} 

\begin{document}

	
\newtheorem{definition}{Definición}[section]
\newtheorem{proposition}{Proposición}[section]
\newtheorem{lemma}{Lema}[section]
\newtheorem{corollary}{Corolario}[section]
\newtheorem{theorem}{Teorema}[section]
\newtheorem{example}{Ejemplo}[section]
\newtheorem{remark}{Nota}[section]

\thispagestyle{empty}
\BgThispage
{
\vspace*{\fill}

\sffamily
\centering
\Large
\vspace*{\fill}

\textbf{{\huge Triangulación de superficies\\ topológicas: Teorema de Radó}}

\vspace{3.5cm}

\textbf{{\LARGE Jesús Bueno Urbano}}

\vspace{0.5cm}

\textbf{Grado en Matemáticas}

\vspace{2cm}

\textbf{Profesor: Francisco J. López Fernández\\
Departamento de Geometría y Topología}

\vspace{2cm}

\textbf{Curso 2016/17}

\vspace{3cm}

\begin{minipage}{3cm}
\centering
\includegraphics[width=3cm]{by-nc-sa-eu.png}
\end{minipage}

\vspace*{\fill}
}

\cleardoublepage
\thispagestyle{empty}
\vspace*{\fill}
\epigraph{«Cuando un topólogo es invitado a dar una conferencia, o a escribir unas líneas sobre el significado de la Topología, no es raro que comience hablando de toros y de tazas de café; de superficies y de bandas de Möbius; de botellas de Klein y planos proyectivos; y tal vez coja una cuerda y comience a mostrarnos prácticamente la teoría de nudos. Pero el mismo topólogo, una vez en clase, no dirá nada de eso, y partiendo de un método axiomático, frío y duro como un trozo de acero, nos hablará de entornos, de abiertos, de espacios conexos, de compactificaciones, de redes, etc.»}{{ \em J.M. Rodríguez Amilibia}}
\vspace*{\fill}
\clearpage

\cleardoublepage
\tableofcontents*
\thispagestyle{empty}
\clearpage

\chapter*{Abstract}
\addcontentsline{toc}{section}{\textbf{Abstract}}
\markboth{Abstract}{Abstract}

Topology is one of the youngest areas of Mathematics. Unlike Algebra, Geometry and Number Theory, whose histories sink into old times, its maturity as a relevant mathematical subject does not get achieved until mid 20th century with the research  of Hausdorff, Alexandrov and Sierpinski.

When we talk about maturity, we understand that it was in that time when its axiomatic development was stablished, this is to say, when basic definitions, working areas, etc. were fixed. However, as we will discuss later,   its origins were found in the middle of Modern Era. Roughly speaking, Topology deals with those properties of geometric objects that keep invariant when they are folded, expanded, contracted or simply deformed, in a such a way that no new points  appear or no different points go to the same point. In other words, when the objects differ by an one-to-one transformations  between their points preserving proximity (i.e., a bijective and continuous map whose inverse is continuous as well). This leads  to the modern concept of homeomorphism.

Topology studies the same objects as Geometry, but from a different point of view. Distances or angles are not relevant, not even points alignment. For example, for a topologist a circle (or an ellipse) are the same thing that any closed and simple curve (called Jordan curve). A ball is not distinguishable from a cube, and it is said that the  ball and the cube are homeomorphic; due to this fact, it is possible to get one from the other by a continuous and reversible transformation. From a metric point of view, equivalent norms induce the same topology in a metric space.

The origins of Topology could be settled down in 18th century. It was then when   the study of geometric properties different from measure or length was started. The first mathematicians called this field {\em Analysis situs} or {\em Geometria situs} (literally, Analysis or Geometry of the place).

In general, mathematicians from 17th and 18th centuries devoted few interest to Topology, with the exception of Leonard Euler. In 1736  he published an article entitled {\em Solutio problematis ad geometriam situs pertinentis} about the problem of Könisberg bridges. This work originated a new researching field, later   called in modern literature {\em graph theory}. In this work,  Euler  noticed  he was working with a different kind of Mathematics where classical Geometry properties were not relevant.
\\ Another remarkable progress due to Euler was his famous formula for a polyhedron (Euler characteristic), which appeared in a letter to Goldbach. Although certainly Euler published the details of this formula in several articles, his demonstration has some gaps; for example, he always supposed that the polyhedroms were convex.

  Euler is an isolated case in classical Mathematics due to his concern about the   rigor in  concepts. In fact,  Topology did not reach a modern axiomatic treatment  till 19th century when  B. Riemman studied the topological connection of surfaces in the decade of 1850. Many authors consider Riemman as the father of Topology, adducing that he describes the present idea of topological space and conceived an independent study program. Even so,  his work gets far away from what today we understand as Topology.

In this memoir we will focus on studying some aspects from low dimensional Topology. Classically, one of the more relevant  topological problems, with deep analytics and geometric implications, have been the study of closed curves in plane. This research line has given rise to theorems so important as the Jordan curve Theorem (and its generalization due to Schönflies) or Radó's Theorem.

In Topology, a Jordan curve in the plane is a closed curve  without self-intersections. In other words, a topological copy of the circle in the Euclidean plane. The classic Jordan curve Theorem says that the complement of a Jordan curve in the plane has two connected components, one bounded (interior of the curve) and the other one unbounded (exterior of the curve), each of them having the curve as boundary. This result seems to be obvious for anyone, although after a bit of reflection about its nature we understand that its proof is far away from being simple. The first mathematician who realized   this was B. Bolzano. Doing an historic survey, the first rigorous proof of the Jordan curve Theorem dealt with polygonal curves. Later, the result was generalized to differentiable ones, and finally Camille Jordan (1887) extended it to continuous curves. The original Jordan's proof was controversial due to it contains some technical gaps. It was Schönflies (1924) who revised Jordan's theorem, proving an stronger generalization known as the Jordan-Schönflies Theorem. This result says that every homeomorphism between two Jordan curves can be extended to a global homeomorphism of the whole plane into itself. In other words, Schönflies proved that the natural embedding of the circle in the plane is topologically rigid.

The Jordan-Schönflies is extraordinary useful from the point of view of two-dimensional Topology, not being an audacity to say that is the base of it. To understand it, remember that the main concern of two-dimensional Topology is the study of topological surfaces. One of the main implications of Jordan-Schönflies Theorem has been the understanding of the cell nature of surfaces, which is the key point for  its classification. It was Radó in 1925 who analyzed the basic topological structure of surfaces, proving that they admit a basic cell  decomposition, or equivalently, that they can be triangulated. The relevance of Radó's Theorem  is that it allows to introduce the polygonal representation for compact surfaces, therefore, hence its full classification by topological surgery.

The goal on this memoir has been the proof of Radó's Theorem by using the Jordan-Schönflies Theorem as fundamental tool. This  memoir goes further from the knowledge  in the subject Topology II, and  requires some complements from Algebra II and Topology I. 

For  making it more self-contained, we have thought necessary to make a little introduction including an schematic proof of the Jordan curve Theorem. This language have allowed us to properly state the Jordan-Schönflies Theorem and explain some of its generalizations. The basic bibliography   used in this memoir has been  the article of C. Thomassen, \cite{Thomassen} and for general topological issues the books of W. S. Massey \cite{Massey} and W. T. Tutte \cite{Tutte}.

\chapter*{Introducción}
\addcontentsline{toc}{section}{\textbf{Introducción}}
\markboth{Introducción}{Introducción}
En Topología, una curva de Jordan es una curva cerrada en el plano sin autointersecciones. En definitiva, una copia topológica de la circunferencia en el plano euclidiano. El teorema clásico de la curva de Jordan  afirma que el complemento de una curva de Jordan en el plano tiene dos componentes conexas, una acotada (interior de la curva) y otra no acotada   (exterior de la curva), siendo la curva la frontera topológica común  de ambas componentes. Este resultado resulta intuitivamente obvio para cualquier observador, aunque tras una breve reflexión sobre su naturaleza se comprende  que su prueba dista mucho de ser trivial. El primer matemático que matizó la dificultad implícita en este enunciado fue B. Bolzano. Haciendo un recorrido histórico, la primera demostración rigurosa del teorema de la curva de Jordan conocida se centró en curvas poligonales, luego se generalizó al ambiente de curvas diferenciables, y finalmente   Camille Jordan (1887) la extendió a curvas continuas generales. La prueba original de Jordan no estuvo ajena a la polémica ya que contenía algunas lagunas técnicas. Fue Schönflies (1924) quien  revisó este resultado clásico,  obteniendo una generalización fuerte conocida en la literatura como teorema de Jordan-Schönflies. Este resultado afirma que cualquier homeomorfismo entre   curvas de Jordan (siempre existen tales homeomorfismos ya que   éstas son copias topológicas de la circunferencia) puede ser extendido a un homeomorfismo global del plano en sí mismo. En otras palabras, Schönflies demostró que el embebimiento natural de la circunferencia en al plano es topológicamente rígido. 

El teorema de Jordan-Schönflies es extraordinariamente útil desde el punto de vista de la Topología bidimensional, no siendo un atrevimiento  afirmar que es el pilar sobre el que se sustenta la misma. 
Para comprenderlo recordemos que la Topología en dos dimensiones se dedica fundamentalmente al estudio de las superficies topológicas, esto es, de los espacios topológicos  Hausdorff y localmente euclidianos.  Una de las implicaciones cruciales del teorema de Jordan-Schönflies  ha sido el facilitar la comprensión de la naturaleza celular de las tales objetos, punto clave para su clasificación. Fue Radó quien en 1925  analizó la estructura topológica básica de las superficies, probando que  éstas admiten una descomposición celular elemental, o equivalentemente, que pueden ser  trianguladas. La relevancia del teorema de Radó  reside en que  es la clave para introducir la representación poligonal de las superficies compactas, y por tanto, para  su clasificación completa mediante cirugía topológica.

El objetivo  de este Trabajo Fin de Grado (TFG) ha sido la demostración del teorema de Radó   utilizando el teorema de Jordan-Schönflies como herramienta básica. La memoria que presentamos comprende una ampliación del conocimiento en relación a la asignatura la Topología II del Grado en Matemáticas,   aunque también ha requerido  de algún material complementario de  Álgebra II y Topología I.  

El objeto de mejorar la exposición y hacerla más autocontenida, hemos creído necesario hacer una pequeña introducción incluyendo  una prueba esquemática   del teorema de la curva de Jordan. Este lenguaje nos ha permitido  presentar el teorema de Jordan-Schönflies, junto con algunas de sus generalizaciones básicas, en el contexto más apropiado para su utilización   en la demostración del teorema de Radó. La referencia bibliográfica básica para la realización de este trabajo ha sido el artículo de C. Thomassen, \cite{Thomassen}  mientras que para cuestiones relativas a la Topología general hemos consultado los libros de W. S. Massey \cite{Massey} y W. T. Tutte. \cite{Tutte}

\newpage

\chapter{Preliminares}

En este capítulo presentaremos las herramientas básicas para el desarrollo del resultado principal de este Trabajo Fin de Grado: El teorema de Radó. Para ello necesitaremos el concepto de curva de Jordan, amén del teorema del mismo nombre y de su generalización, el teorema de Jordan-Schönflies. En la siguiente sección introduciremos el lenguaje que necesitaremos para la demostración de estos resultados, enfatizando el teorema de la curva de Jordan. El teorema de Jordan-Schönflies se enunciará sin demostración porque su tratamiento queda lejos de las competencias dadas a este trabajo.

\section{Teorema de la curva de Jordan}

Comencemos con la idea básica de una curva cerrada en el plano que sin autointersecciones, esto es, una circunferencia.

Es fácil ver   que  ésta divide en dos partes el plano. Como muchos otros resultados en Matemáticas que a simple vista parecen sencillos, probar que toda curva cerrada sin autointersecciones divide en dos partes el plano no es tarea fácil.

Este será el resultado que desarrollaremos en éste apartado, para el que necesitaremos del lenguaje topológico apropiado que introducimos a continuación.

\subsection{Conceptos básicos de Topología}

\begin{definition}
	Un espacio topológico es un par ($X$,$\tau$) donde
	\begin{enumerate}
		\item  $X$ es un conjunto no vacío.
		\item $\tau \subseteq \mathit{P}(X)$
	\end{enumerate}
	satisfaciendo:
	\begin{enumerate}
		\item $\emptyset \text{,} X \in \tau$
		\item Si $O_1 \text{,} O_2 \in \tau \Longrightarrow O_1 \bigcap O_2 \in \tau$
		\item Si $\{ O_\lambda \text{, } \lambda \in \Lambda \} \subseteq \tau \Longrightarrow \bigcup_{\lambda \in \Lambda} O_\lambda \in \tau$
	\end{enumerate}
\end{definition}


\begin{definition}
	Sean $X$ e $Y$ dos espacios topológicos y $f$ una función de $X$ en $Y$; $f$ se dice un homeomorfismo si, y sólo si, cumple las siguientes condiciones:
	\begin{enumerate}
		\item $f$ es biyectiva.
		\item $f$ es continua.
		\item La inversa de $f$ es continua.
	\end{enumerate}
\end{definition}

Si $f : X \rightarrow Y$ es un homeomorfismo, $X$ se dice homeomorfo a $Y$. Además conservan los llamados invariantes topológicos.

Ejemplos claros de homeomorfismos:

\begin{enumerate}
	\item La circunferencia menos un punto es homeomorfa a la recta real. El homeomorfismo definido sería la conocida como proyección estereográfica.
	\item Un cono sin el vértice es homeomorfo a un cilindro.
\end{enumerate}

\begin{definition}
	Un arco en un espacio topológico ($X$,$\tau$) es un aplicación continua $\alpha: I\subset \mathbb{R} \rightarrow (X,\tau)$, donde $I = [a,b] \hspace{0.2cm} \text{, con } a \textless b$, dotado de la topología euclídea.
\end{definition}

Al conjunto $\gamma = \{ \alpha(t) : t \in I \}$ se le llamará traza del arco $\alpha$.

Éste concepto se extiende de forma natural a arcos abiertos definidos en todo $\mathbb{R}$, aunque nosotros nos centraremos a curvas definidas en intervalos de la forma $[a,b]$ como en la definición. Los puntos  $\alpha(a)$ y $\alpha(b)$ se referirán como el inicio y el final de la arco $\alpha$, respectivamente, y también se dirá que éste une $\alpha(a)$ y $\alpha(b)$.

\begin{definition}
	Un arco $\alpha$ se dirá cerrado si sus extremos coinciden, es decir, si $\alpha(a) = \alpha(b)$.
\end{definition}

\begin{definition}
	Una arco $\alpha\colon [a,b]\to X$ se dirá de Jordan si no se autointersecta a sí mismo, esto es:\[\ \alpha(t_1) \neq \alpha(t_2) \hspace{0.5cm}\forall \hspace{0.1cm}t_1,t_2 \in [a,b] \]
En otras palabras, $\alpha$ es de Jordan  si la aplicación $\alpha$ es inyectiva.
\end{definition}



\begin{definition}
	Una curva $\alpha : [a,b] \rightarrow X$ es de Jordan si es un arco cerrado tal que $\alpha \vert_{[a,b[}$ es inyectiva. Equivalentemente, una aplicación continua $\alpha : \mathbb{S}^1 \rightarrow X$ sin autointersecciones.
\end{definition}

\begin{definition}
	Decimos que un espacio topológico ($X$,$\tau$) es arcoconexo, si cualesquiera dos puntos de $X$ están unidos por un arco.
\end{definition}

\begin{definition}
	Un arco poligonal simple (o una curva de Jordan poligonal, si es cerrado) en $\mathbb{R}^2$ con la topología euclídea es un arco simple (o curva de Jordan, si es cerrado) cuya traza es  unión finita de segmentos.
\end{definition}

\begin{lemma}
	Sea $\Omega$ un conjunto abierto conexo del plano, entonces cualesquiera dos puntos en $\Omega$ se pueden unir por un arco poligonal simple en $\Omega$. 
\end{lemma}

\begin{definition}
	Una componente conexa de un espacio topológico es un subespacio topológico conexo maximal. Un espacio topológico se dice localmente conexo (arcoconexo) si todo punto admite una base de entornos conexos (arcoconexos). 
\end{definition}
\begin{lemma}
Todo espacio topológico conexo y localmente arcoconexo es arcoconexo. Como todo abierto del plano es localmente arcoconexo (de hecho, localmente {\em poligonalmente conexo}), las componentes conexas de un abierto del plano son poligonalmente conexas. 
\end{lemma}

\subsection{Teoría básica de grafos}

Una vez presentado el lenguaje topológico necesario para el desarrollo de nuestro tema, introduciremos a continuación la teoría básica de grafos. Ésta nos ayudará a demostrar el teorema de la curva de Jordan, meta principal de este apartado.

\begin{definition}
	Un grafo G es la unión de dos conjuntos disjuntos $V(G)$ y $E(G)$, llamado vértices y aristas respectivamente, tal que para cada arista existen asociados dos vértices distintos $x$ e $y$, llamados extremos de la arista.
\end{definition}



\begin{definition}
	Diremos que dos vértices $u$ y $v$ de un grafo $G$ son adyacentes si, y sólo si, están unidos por una arista.
\end{definition}

A partir de ahora  nos restringiremos a grafos  satisfaciendo la siguiente propiedad: 
\begin{quote} {\em Si dos vértices $x$ e $y$   son adyacentes entonces hay una única arista que los une, que denotaremos $xy$.}
\end{quote}
Por tanto no cabe confusión al nombrar las aristas en función de sus vértices.

\begin{definition}
	Se define un isomorfismo de grafos $G$ y $H$ como una función biyectiva $f : V(G) \rightarrow V(H)$ que preserva la relación de adyacencia. Es decir, un par de vértices $u$ y $v$ de $G$ son adyacentes si, y sólo si, $f(u)$ y $f(v)$ son adyacentes en $H$.
\end{definition}

Por ejemplo los siguientes grafos son isomorfos:
\[\]
\begin{tikzpicture}[y=.3cm, x=.3cm,font=\normalsize]
\draw (0,0) -- (10,10);
\draw (0,0) -- (0,10);
\draw (0,0) -- (10,0);
\draw (10,0) -- (10,10);
\draw (10,0) -- (0,10);
\draw (0,10) -- (10,10);

\filldraw[fill=black!40,draw=black!80] (0,0) circle (3pt)    node[anchor=north] {$v_1$};
\filldraw[fill=black!40,draw=black!80] (0,10) circle (3pt)    node[anchor=south] {$v_2$};
\filldraw[fill=black!40,draw=black!80] (10,10) circle (3pt)    node[anchor=south] {$v_3$};
\filldraw[fill=black!40,draw=black!80] (10,0) circle (3pt)    node[anchor=north] {$v_4$};

\draw (20,0) -- (25,10);
\draw (20,0) -- (25,5);
\draw (20,0) -- (30,0);
\draw (30,0) -- (25,10);
\draw (30,0) -- (25,5);
\draw (25,10) -- (25,5);

\filldraw[fill=black!40,draw=black!80] (20,0) circle (3pt)    node[anchor=north] {$w_1$};
\filldraw[fill=black!40,draw=black!80] (25,10) circle (3pt)    node[anchor=south] {$w_2$};
\filldraw[fill=black!40,draw=black!80] (30,0) circle (3pt)    node[anchor=north] {$w_3$};
\filldraw[fill=black!40,draw=black!80] (25,5) circle (3pt)    node[anchor=north] {$w_4$};
\end{tikzpicture}
\[\]
	El isomorfismo se define para como el único que lleva $v_{i}$ en $w_{i}$ $\forall i = 1,2,3,4$.

\begin{definition}
Un camino es un grafo con vértices $v_{1},...,v_{n}$ y aristas $v_{1},v_{2},...,v_{n-1}v_{n}$. Denotaremos un camino por $v_{1}v_{2}...v_{n}$.

Si además $n \geq 3$ y añadimos una arista $v_{n}v_{1}$ a un camino dado obtenemos un camino cerrado al que llamaremos ciclo y denotaremos por $C_n$.
\end{definition}

	A continuación tenemos un ejemplo sencillo de un camino con 4 vértices y un ciclo con 4 vértices también, a los que denominaremos $v_{1}v_{2}v_{3}v_{4}$ y $C_4$ por la notación que hemos convenido:
\[\]
\begin{tikzpicture}[y=.3cm, x=.3cm,font=\normalsize]
\draw (0,0) -- (0,10);
\draw (10,0) -- (10,10);
\draw (10,10) -- (0,10);

\filldraw[fill=black!40,draw=black!80] (0,0) circle (3pt)    node[anchor=north] {$v_1$};
\filldraw[fill=black!40,draw=black!80] (0,10) circle (3pt)    node[anchor=south] {$v_2$};
\filldraw[fill=black!40,draw=black!80] (10,10) circle (3pt)    node[anchor=south] {$v_3$};
\filldraw[fill=black!40,draw=black!80] (10,0) circle (3pt)    node[anchor=north] {$v_4$};

\node [above] at (5,10) {$v_{2}v_{3}$};
\node [right] at (10,5) {$v_{3}v_{4}$};
\node [left] at (0,5) {$v_{1}v_{2}$};

\draw (20,0) -- (20,10);
\draw (30,0) -- (30,10);
\draw (30,10) -- (20,10);
\draw (20,0) -- (30,0);

\filldraw[fill=black!40,draw=black!80] (20,0) circle (3pt)    node[anchor=north] {$v_1$};
\filldraw[fill=black!40,draw=black!80] (20,10) circle (3pt)    node[anchor=south] {$v_2$};
\filldraw[fill=black!40,draw=black!80] (30,10) circle (3pt)    node[anchor=south] {$v_3$};
\filldraw[fill=black!40,draw=black!80] (30,0) circle (3pt)    node[anchor=north] {$v_4$};

\node [above] at (25,10) {$v_{2}v_{3}$};
\node [right] at (30,5) {$v_{3}v_{4}$};
\node [left] at (20,5) {$v_{1}v_{2}$};
\node [below] at (25,0) {$v_{4}v_{1}$};
\end{tikzpicture}
\[\]
	Si $G$ es un grafo y $A \subseteq V(G) \cup E(G)$, entonces $G - A$ es el grafo obtenido de $G$ al eliminar todos los vértices y aristas que pertenezcan a $A$,  y todas las aristas  de $G$ incidentes en un vértice de $A$ (aunque no pertenezcan a  $A$).

	Veamos un ejemplo gráfico:
\[\]
\begin{tikzpicture}[y=.3cm, x=.3cm,font=\normalsize]
\draw (0,0) -- (10,10);
\draw (0,0) -- (0,10);
\draw (0,0) -- (10,0);
\draw (10,0) -- (10,10);
\draw (10,0) -- (0,10);
\draw (0,10) -- (10,10);

\filldraw[fill=black!40,draw=black!80] (0,0) circle (3pt)    node[anchor=north] {$v_1$};
\filldraw[fill=black!40,draw=black!80] (0,10) circle (3pt)    node[anchor=south] {$v_2$};
\filldraw[fill=black!40,draw=black!80] (10,10) circle (3pt)    node[anchor=south] {$v_3$};
\filldraw[fill=black!40,draw=black!80] (10,0) circle (3pt)    node[anchor=north] {$v_4$};


\draw (15,0) -- (25,10);

\filldraw[fill=black!40,draw=black!80] (15,0) circle (3pt)    node[anchor=north] {$v_1$};

\draw (40,0) -- (40,10);
\draw (40,0) -- (30,10);
\draw (30,10) -- (40,10);

\filldraw[fill=black!40,draw=black!80] (30,10) circle (3pt)    node[anchor=south] {$v_2$};
\filldraw[fill=black!40,draw=black!80] (40,10) circle (3pt)    node[anchor=south] {$v_3$};
\filldraw[fill=black!40,draw=black!80] (40,0) circle (3pt)    node[anchor=north] {$v_4$};
\end{tikzpicture}
\[\]
En este ejemplo tan simple, el grafo de la izquierda sería nuestro grafo $G$, la figura central es lo que hemos llamado $A$ y, finalmente, el grafo de la derecha sería $G-A$.

\begin{definition}
	Decimos que un grafo $G$ es conexo si para cada par de vértices $u$ y $v$ en $G$ se puede encontrar un camino en $G$ que los una.
\end{definition}

\begin{definition}
	Decimos que un grafo $G$ es biconexo si es conexo y para vértice $v$ en $G$, $G - \{v\}$ (que también podremos denotar como $G - v$) es conexo.
\end{definition}

\begin{definition}
Un grafo $G$ se dice que puede ser embebido en un espacio topológico $X$ si los vértices de $G$ pueden ser representados por puntos distintos de $X$ y cada arista de $G$ puede ser representada como un arco de Jordan en X uniendo los puntos que representan a sus vértices, de  modo que dos aristas distintas se cortan a lo más un extremo en común.
\end{definition}

\begin{definition}
Los  grafos abstractos que pueden ser embebidos en $\mathbb{R}^2$ se dirán planos.
\end{definition}

En muchas ocasiones no haremos distinción entre un grafo abstracto plano y el grafo euclidiano asociado   a su representación (o dibujo) en $\mathbb{R}^2$ mediante puntos y arcos. Veamos algunos ejemplos:

\[\]
\begin{tikzpicture}[y=.3cm, x=.3cm,font=\normalsize]

\draw (0,0) -- (5,10);
\draw (0,0) -- (10,0);
\draw (10,0) -- (5,10);
\draw (10,0) -- (0,5);
\draw (0,5) -- (10,5);
\draw (0,5) -- (5,10);
\draw (10,5) -- (5,10);

\filldraw[fill=black!40,draw=black!80] (0,0) circle (3pt)    node[anchor=north] {$v_1$};
\filldraw[fill=black!40,draw=black!80] (5,10) circle (3pt)    node[anchor=south] {$v_2$};
\filldraw[fill=black!40,draw=black!80] (10,0) circle (3pt)    node[anchor=north] {$v_3$};
\filldraw[fill=black!40,draw=black!80] (0,5) circle (3pt)    node[anchor=east] {$v_4$};
\filldraw[fill=black!40,draw=black!80] (10,5) circle (3pt)    node[anchor=west] {$v_5$};


\draw (15,0) -- (20,10);
\draw (15,0) -- (25,0);
\draw (25,0) -- (20,10);
\draw (25,0) -- (30,5);
\draw (30,5) -- (25,5);
\draw (30,5) -- (20,10);
\draw (25,5) -- (20,10);

\filldraw[fill=black!40,draw=black!80] (15,0) circle (3pt)    node[anchor=north] {$w_1$};
\filldraw[fill=black!40,draw=black!80] (20,10) circle (3pt)    node[anchor=south] {$w_2$};
\filldraw[fill=black!40,draw=black!80] (25,0) circle (3pt)    node[anchor=north] {$w_3$};
\filldraw[fill=black!40,draw=black!80] (30,5) circle (3pt)    node[anchor=west] {$w_4$};
\filldraw[fill=black!40,draw=black!80] (25,5) circle (3pt)    node[anchor=north] {$w_5$};
\end{tikzpicture}
\[\]

A simple vista, la figura de la izquierda no parecería un grafo plano, pero definiendo un isomorfismo vemos que podemos transformarlo en el grafo de la figura de la derecha que claramente es plano.

\begin{lemma}\label{lema22}
	Si $G$ es un grafo plano, entonces $G$ puede embeberse en el plano de tal forma que todas sus aristas son arcos poligonales simples.
\end{lemma}

\begin{definition}
	Una subdivisión de un grafo $G$ es un grafo obtenido a partir de $G$ en el que varios, o todos, de sus aristas son reemplazados por caminos con los mismos extremos.
\end{definition}

Ejemplo:

\[\]
\begin{tikzpicture}[y=.3cm, x=.3cm,font=\normalsize]

\draw (0,0) -- (10,0);

\filldraw[fill=black!40,draw=black!80] (0,0) circle (3pt)    node[anchor=north] {$x$};
\filldraw[fill=black!40,draw=black!80] (10,0) circle (3pt)   node[anchor=north] {$z$};

\draw (15,0) -- (25,0);

\filldraw[fill=black!40,draw=black!80] (15,0) circle (3pt)   node[anchor=north] {$x$};
\filldraw[fill=black!40,draw=black!80] (25,0) circle (3pt)   node[anchor=north] {$z$};
\filldraw[fill=black!40,draw=black!80] (20,0) circle (3pt)   node[anchor=north] {$y$};
\end{tikzpicture}
\[\]

Hemos creado una subdivisión de la arista $xz$ en el camino $xyz$.
	
\begin{definition}
	Sea un grafo $G$ se dice completo si cada par de vértices $u$ y $v$ están conectados por una arista. Si el grafo $G$ completo tiene n vértices lo denominaremos $K_n$.
\end{definition}

\begin{definition}
	Un grafo $G$ se dice bipartito si su conjunto de vértices $V(G)$ es unión de dos subconjuntos disjuntos $V_1$ y $V_2$, de manera que las aristas del grafo conectan vértices $V_1$ y $V_2$ pero nunca    vértices de un mismo subconjunto ($V_1$ ó $V_2$). Es decir, existen $V_1,V_2 \subset V(G)$ cumpliendo:
	\begin{itemize}
		\item $V_1 \cup V_2 = V(G)$
		\item $V_1 \cap V_2 = \emptyset$
		\item $\forall v_{1},v_{2} \in V_j \hspace{0.1cm}  \hspace{0.1cm} \nexists e_{1},e_{2} \in E(G)$ tal que $e_{1}$ tenga como extremos a $v_{1}$ y $v_{2}$, $j=1,2$.
	\end{itemize}
\end{definition}

Ejemplo:

\begin{tikzpicture}[y=.3cm, x=.3cm,font=\normalsize]
	\draw (0,10) -- (4,0);
	\draw (0,10) -- (20,0);
	\draw (8,10) -- (4,0);
	\draw (8,10) -- (12,0);
	\draw (16,10) -- (12,0);
	\draw (16,10) -- (20,0);

	\filldraw[fill=black!40,draw=black!80] (0,10) circle (3pt)   node[anchor=south] {$v_1$};
	\filldraw[fill=black!40,draw=black!80] (8,10) circle (3pt)   node[anchor=south] {$v_3$};
	\filldraw[fill=black!40,draw=black!80] (16,10) circle (3pt)  node[anchor=south] {$v_5$};
	\filldraw[fill=black!40,draw=black!80] (4,0) circle (3pt)    node[anchor=north] {$v_2$};
	\filldraw[fill=black!40,draw=black!80] (12,0) circle (3pt)   node[anchor=north] {$v_4$};
	\filldraw[fill=black!40,draw=black!80] (20,0) circle (3pt)   node[anchor=north] {$v_6$};

\end{tikzpicture}

\begin{definition}
	Un grafo bipartito $G$ con  conjuntos de vértices destacados  $V_1$ y $V_2$ se dirá completo si  $\forall v_1 \in V_1 \text{,} \;\forall v_2 \in V_2 $, existe una arista que los conecta. Al grafo $G$ bipartito completo de n vértices en $V_1$ y m vértices en $V_2$ lo denominaremos $K_{n,m}$.
\end{definition}

Ejemplos de grafos completo ($K_5$), bipartito y bipartito completo ($K_{3,3}$):
	
\[\]
\begin{tikzpicture}[y=.3cm, x=.3cm,font=\normalsize]
\draw (2,0) -- (0,6);
\draw (2,0) -- (4,8);
\draw (2,0) -- (8,6);
\draw (2,0) -- (6,0);
\draw (0,6) -- (4,8);
\draw (0,6) -- (8,6);
\draw (0,6) -- (6,0);
\draw (4,8) -- (8,6);
\draw (4,8) -- (6,0);
\draw (8,6) -- (6,0);

\filldraw[fill=black!40,draw=black!80] (2,0) circle (3pt)    node[anchor=north] {$v_1$};
\filldraw[fill=black!40,draw=black!80] (0,6) circle (3pt)    node[anchor=east] {$v_2$};
\filldraw[fill=black!40,draw=black!80] (4,8) circle (3pt)    node[anchor=south] {$v_3$};
\filldraw[fill=black!40,draw=black!80] (8,6) circle (3pt)    node[anchor=west] {$v_4$};
\filldraw[fill=black!40,draw=black!80] (6,0) circle (3pt)    node[anchor=north] {$v_5$};

\draw (15,8) -- (18,0);
\draw (15,8) -- (21,0);
\draw (24,8) -- (18,0);

\filldraw[fill=black!40,draw=black!80] (15,8) circle (3pt)    node[anchor=south] {$v_1$};
\filldraw[fill=black!40,draw=black!80] (24,8) circle (3pt)    node[anchor=south] {$v_2$};
\filldraw[fill=black!40,draw=black!80] (18,0) circle (3pt)    node[anchor=north] {$v_3$};
\filldraw[fill=black!40,draw=black!80] (21,0) circle (3pt)    node[anchor=north] {$v_4$};

\draw (31,8) -- (33,0);
\draw (31,8) -- (37,0);
\draw (31,8) -- (41,0);
\draw (35,8) -- (33,0);
\draw (35,8) -- (37,0);
\draw (35,8) -- (41,0);
\draw (39,8) -- (33,0);
\draw (39,8) -- (37,0);
\draw (39,8) -- (41,0);

\filldraw[fill=black!40,draw=black!80] (31,8) circle (3pt)    node[anchor=south] {$v_1$};
\filldraw[fill=black!40,draw=black!80] (35,8) circle (3pt)    node[anchor=south] {$v_2$};
\filldraw[fill=black!40,draw=black!80] (39,8) circle (3pt)    node[anchor=south] {$v_3$};
\filldraw[fill=black!40,draw=black!80] (33,0) circle (3pt)    node[anchor=north] {$v_4$};
\filldraw[fill=black!40,draw=black!80] (37,0) circle (3pt)    node[anchor=north] {$v_5$};
\filldraw[fill=black!40,draw=black!80] (41,0) circle (3pt)    node[anchor=north] {$v_6$};
\end{tikzpicture}
\[\]

\begin{theorem}[\textbf{Teorema de Kuratowski}]
	Un grafo $G$ es plano si y sólo si no contiene ningún subgrafo que sea una subdivisión de $K_5$ ó $K_{3,3}$.
\end{theorem}
	
Por ejemplo: ¿El grafo $K_6$ es plano?
\[\]
\begin{tikzpicture}[y=.3cm, x=.3cm,font=\normalsize]
\draw (15,0) -- (10,5);
\draw (15,0) -- (15,10);
\draw (15,0) -- (20,10);
\draw (15,0) -- (25,5);
\draw (15,0) -- (20,0);
\draw (10,5) -- (15,10);
\draw (10,5) -- (20,10);
\draw (10,5) -- (25,5);
\draw (10,5) -- (20,0);
\draw (15,10) -- (20,10);
\draw (15,10) -- (25,5);
\draw (15,10) -- (20,0);
\draw (20,10) -- (25,5);
\draw (20,10) -- (20,0);
\draw (25,5) -- (20,0);

\filldraw[fill=black!40,draw=black!80] (15,0) circle (3pt)    node[anchor=north] {$v_1$};
\filldraw[fill=black!40,draw=black!80] (10,5) circle (3pt)    node[anchor=east] {$v_2$};
\filldraw[fill=black!40,draw=black!80] (15,10) circle (3pt)    node[anchor=south] {$v_3$};
\filldraw[fill=black!40,draw=black!80] (20,10) circle (3pt)    node[anchor=south] {$v_4$};
\filldraw[fill=black!40,draw=black!80] (25,5) circle (3pt)    node[anchor=west] {$v_5$};
\filldraw[fill=black!40,draw=black!80] (20,0) circle (3pt)    node[anchor=north] {$v_6$};
\end{tikzpicture}
\[\]
	
	Por el teorema de Kuratowski, $K_{6}$ no es plano. En efecto, si eliminamos uno solo de sus vértices, y por ende las aristas que inciden sobre éste, obtenemos el grafo $K_5$.
	
	Una demostración muy simple y corta del teorema de Kuratowski se puede ver en \cite{Kuratowski}. No obstante nosotros sólo necesitaremos el hecho de que $K_{3,3}$ no es plano.
	Este resultado es la base de la demostración del teorema de la curva de Jordan y además no es complicado de demostrar, por lo tanto convendría pararse un momento para reflexionar sobre él. Para su demostración necesitamos el siguiente resultado.
	
\begin{lemma}
	Sea $C$ una curva poligonal de Jordan en el plano, entonces $\mathbb{R}^2 \backslash C$ tiene exactamente dos regiones con $C$ como frontera.
\end{lemma}

	A estas dos regiones las llamaremos interior y exterior de $C$ y las denotaremos por Int($C$) y Ext($C$) respectivamente. Además se tiene:
\[
\overline{Int(C)} = Int(C) \cup C
\]
\[
\overline{Ext(C)} = Ext(C) \cup C
\]

Ahora vamos a extender este resultado.

\begin{lemma}\label{lema24}
	Sea $C$ una curva poligonal simple y cerrada en el plano y $P$ un arco poligonal simple en $\overline{Ext(C)}$ tal que $P$ une $p$ y $q$ en $C$ y no tiene otros puntos en común con $C$. Sean $P_1$ y $P_2$ los dos arcos de $C$ con interiores disjuntos conectando $p$ a $q$: $C=P_1\cup P_2$, $P_1\cap P_2=\{p,q\}$. Entonces $\mathbb{R}^2 \backslash (C \cup P)$ tiene exactamente tres regiones con $C$,$P_{1} \cup P$ y $P_{2} \cup P$ como frontera, respectivamente.
	
El enunciado análogo es cierto si $P$ está contenido en  $\overline{Int(C)}$ y es disjunto con $C$ salvo sus extremos.
\end{lemma}

	Este lema implica que si $r$ y $s$ son puntos de $P_1 \backslash \{ p,q \}$ y $P_2 \backslash \{ p,q \}$, respectivamente, entonces no es posible unir $r$ con $s$  por arco poligonal simple en $\overline{Int(C)}$ sin cortar a $P$.

\begin{theorem}\label{lema25}
	$K_{3,3}$ no es plano.
\end{theorem}
\begin{proof}
Supongamos ahora que $K_{3,3}$ es  plano, y en virtud del lema \ref{lema22} entendámoslo dibujado en el plano de forma que sus aristas sean arcos poligonales simples. Llamemos como es habitual $V_1=\{v_1,v_2,v_3\}$ y $V_2=\{v_4,v_5,v_6\}$ a los dos conjuntos de vértices destacados de $K_{3,3}$.
Consideremos el ciclo poligonal $C_6$ en $\mathbb{R}^2$ definido sobre el dibujo embebido de  $K_{3,3}$  por los vértices $v_{1}v_{2}v_{3}v_{4}v_{5}v_{6}$. Destaquemos las aristas $v_{1}v_{4}$, $v_{2}v_{5}$ y $v_{3}v_{6}$ de $K_{3,3}$, obviamente disjuntas entre si y disjuntas con $C_6$,  salvo los vértices extremos contenidos en $C_6$. Claramente $C_6$ sería una curva poligonal de Jordan, y por el Lema anterior cada dos de las tres aristas destacadas han de estar contenidas en componentes conexas distintas de $\mathbb{R}^2\setminus C$ (excepto sus puntos extremos), ya que éstas no se cortan.  Esto contradice que $\mathbb{R}^2\setminus C$ tiene sólo dos componentes conexas.
\end{proof}

A continuación mostramos  los grafos que hemos usado en la demostración.

\begin{tikzpicture}[y=.3cm, x=0.3cm,font=\normalsize]
\draw (0,10) -- (4,0);
\draw (0,10) -- (20,0);
\draw (8,10) -- (4,0);
\draw (8,10) -- (12,0);
\draw (16,10) -- (12,0);
\draw (16,10) -- (20,0);

\filldraw[fill=black!40,draw=black!80] (0,10) circle (3pt)    node[anchor=south] {$v_1$};
\filldraw[fill=black!40,draw=black!80] (8,10) circle (3pt)    node[anchor=south] {$v_3$};
\filldraw[fill=black!40,draw=black!80] (16,10) circle (3pt)    node[anchor=south] {$v_5$};
\filldraw[fill=black!40,draw=black!80] (4,0) circle (3pt)    node[anchor=north] {$v_2$};
\filldraw[fill=black!40,draw=black!80] (12,0) circle (3pt)    node[anchor=north] {$v_4$};
\filldraw[fill=black!40,draw=black!80] (20,0) circle (3pt)    node[anchor=north] {$v_6$};

\draw (22,10) -- (34,0);
\draw (30,10) -- (42,0);
\draw (38,10) -- (26,0);

\filldraw[fill=black!40,draw=black!80] (22,10) circle (3pt)    node[anchor=south] {$v_1$};
\filldraw[fill=black!40,draw=black!80] (30,10) circle (3pt)    node[anchor=south] {$v_3$};
\filldraw[fill=black!40,draw=black!80] (38,10) circle (3pt)    node[anchor=south] {$v_5$};
\filldraw[fill=black!40,draw=black!80] (26,0) circle (3pt)    node[anchor=north] {$v_2$};
\filldraw[fill=black!40,draw=black!80] (34,0) circle (3pt)    node[anchor=north] {$v_4$};
\filldraw[fill=black!40,draw=black!80] (42,0) circle (3pt)    node[anchor=north] {$v_6$};
\end{tikzpicture}

Ahora ya estamos listos para llegar al objetivo de éste apartado: El teorema de la curva de Jordan.

\subsection{Teorema de la curva de Jordan: Demostración}

Para la demostración del teorema de la curva de Jordan necesitaremos, a parte del hecho ya enunciado de que $K_{3,3}$ no es plano, dos resultados que nos serán de utilidad.

\begin{lemma}\label{proposicion26}
	Si $C$ es una curva de Jordan, entonces $\mathbb{R}^{2} \backslash C$ es disconexo.
\end{lemma}
	
\begin{proposition}\label{proposicion211}
	Si $P$ es un arco de poligonal simple en el plano, entonces $\mathbb{R}^2 \backslash P$ es conexo.
\end{proposition}

\begin{definition}
	Sea $C$ un subconjunto cerrado del plano y $\Omega$ una región de $\mathbb{R}^2 \backslash C$. Un punto $p \in C$ es accesible desde $\Omega$ si para algún (y por lo tanto para cada) punto $q \in \Omega$ existe un arco poligonal simple de $q$ a $p$ teniendo sólo el punto $p$ en común con $C$.
\end{definition}

Si $C$ es una curva de Jordan, $p\in C$ y $\Omega$ es una región de  $\mathbb{R}^2 \backslash C$,  no está claro que $p$ tenga que ser accesible desde $\Omega$. Sin embargo, tenemos el siguiente resultado.

\begin{lemma}
Si $C$ es una curva de Jordan en el plano y $\Omega$ es una región de  $\mathbb{R}^2 \backslash C$, el conjunto de puntos de $C$  accesibles desde $\Omega$  es denso en $C$.
\end{lemma}
\begin{proof} En efecto, si $P$ es cualquier arco de $C$ que contiene a $p\in C$, entonces la proposición \ref{proposicion211} implica que $\mathbb{R}^2 \backslash (\overline{C \backslash P})$ es conexo y por tanto contiene un arco poligonal simple $P'$ conectando $p$ con un punto  $q$ de cualquiera región de $\mathbb{R}^2 \backslash C$ distinta de $\Omega$ (que sabemos existe ya que $\mathbb{R}^2 \backslash C$  es disconexo). Necesariamente  $P'$ ha de cortar a $C$ en algún punto de $P$ que será accesible desde $\Omega$, y como $P$ puede ser  arbitrariamente pequeño  concluimos que los puntos de $C$ accesibles desde $\Omega$ son densos en $C$. 
\end{proof}
Una consecuencia inmediata de este lema es que si $C$ es una curva Jordan  en el plano, $C$ es la frontera de cualquiera de las componentes conexas de $\mathbb{R}^2\setminus C$.
Podemos ahora demostrar el teorema buscado.

\begin{theorem}[\textbf{Teorema de la curva de Jordan}]
	Sea $C$ una curva de Jordan $\Rightarrow$ $\mathbb{R}^2 \backslash C$ tiene exactamente dos regiones, ambas con $C$ como frontera.
\end{theorem}

\begin{proof}
Realizaremos ésta demostración por reducción al absurdo, y asumamos oor el Lema \ref{proposicion26}  que $\mathbb{R}^2 \backslash C$ tiene más de dos componentes conexas. Fijemos  tres componentes conexas distintas $\Omega_1$, $\Omega_2$ y $\Omega_3$ de $\mathbb{R}^2 \backslash C$ y elijamos $q_j\in \Omega_j$, $j=1,2,3$.
Sean $Q_1$, $Q_2$, $Q_3$ subarcos disjuntos dos a dos de $C$. 	 Por el lema anterior, $Q_j$ contiene puntos accesibles desde $\Omega_i$, por lo que $\Omega_i$ contiene un arco poligonal simple $P_{i,j}$ con punto inicial $q_i$ y final en un punto de $Q_j$,  $\forall i,j =$ 1, 2, 3. Salvo rectificar el dibujo de los arcos, podemos asumir que $P_{i,j} \cap P_{i,j'} = {q_i}$ $j \neq j'$. Además es claro que $P_{i,j} \cap P_{i',j'} = \emptyset$ por estar en regiones distintas de $\mathbb{R}^2 \backslash C$.
Añadiendo a los arcos $P_{i,j}$ convenientes trozos de arco  dibujados sobre $C$,  los arcos de Jordan resultantes (también denotados $P_{i,j}$) definen un grafo plano isomorfo a $K_{3,3}$, lo que contradice el Teorema \ref{lema25}.

Por tanto, $\mathbb{R}^2 \backslash C$ tiene exactamente dos regiones,  que llamaremos  $Ext(C)$ (no acotada) e $Int(C)$ (acotada), con frontera común $C$.
\end{proof}

Como nota final, sólo decir que el teorema de la curva de Jordan es un caso particular del teorema de Jordan-Schönflies que enunciaremos en la siguiente sección.

\section{Teorema de Jordan-Schönflies}

En ésta segunda sección vamos a comentar algunas implicaciones topológicas del teorema de Jordan-Schönflies, una generalización fuerte del teorema de la curva de Jordan. Este resultado  se enunciará si demostración y que su tratamiento excede el ámbito de este trabajo. 
Comenzaremos extendiendo el lema \ref{lema24}:

\begin{lemma}
	Sea $C$ una curva de Jordan y $P$ un arco   simple en $Int(C)$ tal que $P$ una $p,q \in C$ y no tiene otro punto en común con $C$. Sean $P_1$ y $P_2$ los dos arcos en $C$ que unen $p$ y $q$. Entonces $\mathbb{R}^2 \backslash (C \cup P)$ tiene exactamente tres regiones cuyas fronteras son $C$, $P\cup P_1$ y $P\cup P_2$, respectivamente.
\end{lemma}

	La generalización que se ha hecho en éste lema es que hemos pasado de poligonales simples cerradas a curvas de Jordan.
	
	\begin{definition}
Si $S$ es un conjunto, entonces $\vert S \vert$ denotará su cardinal.
\end{definition}
El siguiente resultado, corolario inductivo de lema anterior, simplemente expresa que la característica de Euler de la esfera es $2$:
\begin{lemma}
	Si $\Gamma$ es un grafo plano biconexo conteniendo un ciclo $C$ (que de hecho es una curva de Jordan) tal que todas aristas de $\Gamma \backslash C$ son arcos simples en $\overline{Int(C)}$. Entonces $\mathbb{R}^2 \backslash \Gamma$ tiene exactamente $\vert E(\Gamma) \vert - \vert V(\Gamma) \vert + 2$ regiones (llamadas caras de $\Gamma$). Cada una de ellas con un ciclo de $\Gamma$ como frontera.
\end{lemma}

Un uso elaborado del anterior lema  permite demostrar el siguiente resultado, uno de los pilares fundamentales de la Topología plana.

\begin{theorem}[\textbf{Teorema de Jordan-Schönflies}]
	Sea $f$ un homeomorfismo de una curva de Jordan $C$ en otra curva curva de Jordan $C'$. Entonces $f$ puede ser extendido a un homeorfismo $F$ de todo el plano.
\end{theorem}
 

\begin{definition}
	Sea $F$ un conjunto cerrado del plano, decimos que un punto $p \in F$ es curvo-accesible si, para cada punto $q$ que no esté en $F$, existe un arco simple de $q$ a $p$ que sólo tenga $p$ en común con F.
\end{definition}

Como trivialmente todo punto de $\mathbb{S}^1$ es curvo accesible, el teorema de Jordan-Schönflies implica que que todo punto en una curva de Jordan es curvo-accesible. Como consecuencia, y usando el círculo de ideas alrededor del teorema de la curva de Jordan, es posible demostrar el siguiente:

\begin{theorem}
	Sea $F$ un conjunto cerrado en el plano con, al menos, tres puntos curvo-accesibles, entonces $\mathbb{R}^2 / F$ tiene como máximo dos regiones.
\end{theorem}

\begin{definition}
	Si $C$ y $C'$ son curvas de Jordan y $\Gamma$ y $\Gamma'$ son grafos biconexos que consisten en $C$ (respectivamente $C'$) y arcos   simples en $\overline{Int}(C)$ (respectivamente $\overline{Int}(C')$), entonces $\Gamma$ y $\Gamma'$ se dice que son plano-isomorfos si existe un isomorfismo de $\Gamma$ en $\Gamma'$ cumpliendo 
\begin{enumerate}
	\item Un ciclo en $\Gamma$ es la frontera de una cara de $\Gamma$     $\iff$ La imagen del ciclo por el isomorfismo es la frontera de una cara de $\Gamma'$.
	\item La imagen por el isomorfismo del ciclo exterior de $\Gamma$ (véase $C$) es el ciclo exterior de $\Gamma'$ (véase $C'$).
\end{enumerate}	
\end{definition}
El teorema de Jordan-Schönflies admite la siguiente generalización:
\begin{theorem}\label{teorema33}
	Sean $\Gamma$ y $\Gamma'$ grafos planos biconexos tales que $g$ sea un homeomorfismo y un plano-isomorfismo de $\Gamma$ en $\Gamma'$. Entonces $g$ puede ser extendido a un homeomorfismo de todo el plano.
\end{theorem}

\begin{proof}
	Esta demostración la haremos por inducción sobre el número de aristas de $\Gamma$.

	Si $\Gamma$ es un ciclo, entonces el problema se reduce al teorema de Jordan-Schönflies. De otro modo se sigue del lema \ref{lema27} que $\Gamma$ tiene un camino $P$ y un subgrafo biconexo $\Gamma_1$ que contiene al ciclo exterior de $\Gamma$, de tal modo que $\Gamma$ se obtiene a partir de $\Gamma_1$ añadiéndole $P$ en $\overline{Int}(C)$, donde $C$ es la frontera de una de las caras de $\Gamma_1$.

	Ahora aplicamos la hipótesis de inducción a $\Gamma_1$ y después a los dos ciclos de $C \cup P$ conteniendo a $P$.
\end{proof}

\chapter{Teorema de Radó}

Finalmente estamos en condiciones de abordar el resultado principal de este Trabajo Fin de Grado, el teorema de Radó. Primero recordaremos  el concepto de superficie, seguiremos con el concepto de triangulación y por último el enunciado del teorema de Radó.

\begin{definition}
Una superficie $S \neq \emptyset$ es un espacio topológico que cumple que $\forall x \in S \exists V \subseteq S$ abierto con $x \in V$ y un homeomorfismo $\Phi : V \rightarrow U$ con $U$ un abierto de $\mathbb{R}^2$.A cada par $(V,\Phi)$ lo llamaremos carta local o sistema de coordenadas de $S$.
\end{definition}

\begin{remark}
Impondremos que $S$ sea Hausdorff y que cumpla el II Axioma de Numerabilidad.
\end{remark}

Consideremos ahora una colección $X$ de polígonos convexos dos a dos, junto con sus respectivos interiores, en el plano euclídeo tal que todas sus aristas son de longitud uno. Definamos un espacio topológico $S$ identificando cada arista de un polígono de $X$  con una  arista de otro, o del mismo, polígono. Está claro que el espacio identificación $S$, con la topología cociente inducida por la proyección natural $\pi\colon X\to S$,  es compacto. Además $S$ contiene de forma natural como subespacio el grafo $G$ con  vértices   la proyección de las esquinas de los polígonos de $X$ y como aristas la proyección de los lados de los polígonos de $X$.

\begin{lemma}
El espacio topológico $S$ anteriormente definido es una superficie si y sólo si $S$ es conexo y además es localmente homeomorfo a un disco en cada vértice de $G$.
\end{lemma}

\begin{definition}
En la situación anterior, diremos que el grafo $G$ es un embebimiento bicelular en $S$, es decir, que éste es un grafo embebido y además cada una de sus caras es homeomorfa a un disco abierto.
\end{definition}

\begin{definition}
Si todos los polígonos son triángulos, entonces decimos que $G$ es una triangulación de $S$ y que $S$ es una superficie triangulable.
\\
En el caso que estemos ante una triangulación tendremos que asumir que tenemos al menos cuatro triángulos (de otra forma $S$ no sería una superficie).
\end{definition}

Para la demostración del teorema de Radó necesitaremos el siguiente resultado de Teoría de Grafos.
Este lema se podría haber encuadrado perfectamente en la sección que trataba sobre teoría básica de grafos, pero debido a que no se hace uso de él hasta este momento se ha ubicado en este apartado.

\begin{lemma}\label{lema27}
	Si $G$ es un grafo biconexo y $H$ es un subgrafo biconexo de $G$, entonces $G$ puede ser obtenido a partir de $H$ añadiendo caminos de forma sucesiva tal que cada uno de estos caminos una dos vértices distintos del grafo actual y el resto de vértices esté fuera del mismo.
\end{lemma}

\begin{proof}
	Haremos la demostración por inducción sobre el número de aristas de $E(G) \backslash E(H)$.

	Si ese número es cero, tenemos $G = H$. No hay nada que probar. Así que  asumamos que $G \neq H$. Por la hipótesis de inducción, el lema \ref{lema27} es válido cuando el par ($G,H$) es sustituido por otro par ($G',H'$) tal que $E(G') \backslash E(H')$ tiene menos aristas que $E(G) \backslash E(H)$. Ahora sea $H'$ un grafo biconexo maximal  de $G$ conteniendo a $H$. Si $H' \neq H$ aplicamos la hipótesis de inducción sobre ($H',H$) y entonces sobre ($G,H'$). Así que asumimos $H' = H$.

	Como $G$ es conexo, hay una arista $x_{1}x_{2}$ en $E(G) \backslash E(H)$ tal que $x_{1} \in H$. Como $G - x_{1}$ es conexo ($G$ es biconexo) existe un camino $P = x_{2}x_{3}...x_{k}$ tal que $x_{k} \in H$ y $X_i \notin H$ $\forall i = 2,... k-1$. Quizás $k = 2$. Ya que $H \cup P \cup {x_{1}x_{2}}$ es biconexo, tenemos que $G = H \cup P \cup {x_{1}x_{2}}$, lo que acaba la demostración.
\end{proof}

\begin{theorem}[\textbf{Teorema de Radó}]
	Toda superficie $S$ es homeomorfa a una superficie triangulable.
\end{theorem}

\begin{proof}

	Como el interior de un polígono convexo puede ser triangulado por métodos elementales, bastará demostrar que la superficie $S$ dada es homeomorfa a una superficie con un embebimiento bicelular.

	Para cada $p \in S$ sea $D(p)$ un disco abierto en $\mathbb{R}^2$ homeomorfo a un entorno de $p$ en $S$. Nótese que en lugar de definir un homeomorfismo usaremos la misma notación para un punto en $D(p)$ y el correspondiente punto en $S$, esto es,  por conveniencia trataremos $D(p)$ como si fuese un subconjunto de $S$.

	En $D(p)$ dibujamos dos cuadrángulos $Q_{1}(p)$ y $Q_{2}(p)$ de forma que $p \in Int(Q_{1}(p)) \subset Int(Q_{2}(p))$. Como $S$ es  compacta, existen un número finito de puntos $p_{1},...,p_{n}$ tal que $S = \bigcup_{i = 1}^{n} Int(Q_{1}(p_{i}))$.

\begin{figure}[h]
\centering
\begin{minipage}[c]{\textwidth}
\centering
    \includegraphics[width=10.0cm]{pic1.jpg}
\end{minipage}
\end{figure}

	Vistos como subconjuntos en el plano, $D(p_{1}),...,D(p_{n})$ se puede asumir que son disjuntos dos a dos (salvo trasladarlos convenientemente). A partir de aquí vamos a tomar estos subconjuntos fijos en el plano, teniendo siempre en cuenta que se corresponden con subconjuntos en $S$.

	Sin embargo, vamos a modificar el homeomorfismo entre $D(p_i)$ y su correspondiente conjunto en $S$ y considerar nuevos cuadrángulos $Q_{1}(p_i)$ de forma que $Q_{1}(p_{1}),...,Q_{1}(p_{n})$  determinen un embebimiento bicelular en S.

La clave será probar que existen cuadrados  $Q_{1}(p_{1}),...,Q_{1}(p_{n})$  como arriba, esto es con $Int(Q_{1}(p_j)) \subset Int(Q_{2}(p_j))\subset D(p_j)$ para todo $j$ y $S = \bigcup_{i = 1}^{n} Int(Q_{1}(p_{i}))$, de forma que  cada dos   tengan en común un número finito de puntos en común de $S$. 

\begin{figure}[h]
\centering
\begin{minipage}[c]{\textwidth}
\centering
    \includegraphics[width=10.0cm]{pic2.jpg}
\end{minipage}
\end{figure}

	Supongamos, por inducción sobre k, que los $Q_{1}(p_{1}),...,Q_{1}(p_{k-1})$ han sido escogidos de forma que cualquiera de cada dos tienen en común un número finito de puntos de $S$. 

	Ahora nos centramos en la posición relativa de los $Q_{1}(p_j)$ ($1 \leq j \leq k-1$) respecto a $Q_{2}(p_k)$. Un arco $P$ de algún $Q_{1}(p_j)$ ($1 \leq j \leq k-1$) que une dos puntos de $Q_{2}(p_k)$ y que tiene el resto de puntos en $Int(Q_{2}(p_k))$ se dirá un 
 {\em segmento malo}. 
 
Fijemos ahora $Q_{3}(p_k)$ un cuadrado entre $Q_{1}(p_k)$ y $Q_{2}(p_k)$. Diremos que  un segmento malo en $Q_{2}(p_k)$ es {\em muy malo} si interseca a $Q_{3}(p_k)$. Puede haber infinitos segmentos malos. Como los $Q_{1}(p_{j})$, $1\leq j\leq k-1$, son curvas de Jordan y por tanto homeomorfos a $\mathbb{S}^1$, no es difícil comprobar que sólo puede haber una cantidad finita de segmentos muy malos. Veámoslo:

\begin{figure}[h]
\centering
\begin{minipage}[c]{\textwidth}
\centering
    \includegraphics[width=10.0cm]{pic3.jpg}
\end{minipage}
\end{figure}

	Los segmentos muy malos junto con $Q_{2}(p_k)$ determinan un grafo biconexo $\Gamma$. Redibujemos $\Gamma$ dentro de $Q_{2}(p_k)$  para obtener un grafo $\Gamma'$ que sea plano-isomorfo a $\Gamma$ y tal que todas aristas de $\Gamma'$ son arcos poligonales simples. Técnicamente ésto se puede llevar a cabo gracias al lema \ref{lema27}, que si lo recordamos dice que un grafo biconexo puede ser obtenido a partir de un subgrafo biconexo de éste añadiendo caminos, sucesivamente, que unan dos vértices del grafo actual y que los otros vértices no estén en éste.

	Ahora aplicamos el teorema \ref{teorema33} para extender el plano-isomorfismo de de $\Gamma$ a $\Gamma'$ a un homeomorfismo de $\overline{Int(Q_{2}(p_k))}$ manteniendo $Q_{2}(p_k)$ fijado. Esto transforma $Q_{1}(p_k)$ y $Q_{3}(p_k)$ en curvas de Jordan $Q_{1}'$ y $Q_{3}'$ tales que $p_{k} \in Int(Q_{1}') \subset Int(Q_{3}')$.

	Consideremos una poligonal de Jordan $Q_{3}''$ en $Int(Q_{2}(p_k))$ tal que $Q_{1}' \subset Int(Q_{3}'')$ y además $Q_{3}''$ no interseca segmentos malos exceptuando los muy malos (que ahora son arcos poligonales simples).$^{1}$
	
\begin{figure}[h]
\centering
\begin{minipage}[c]{\textwidth}
\centering
    \includegraphics[width=10.0cm]{pic4.jpg}
\end{minipage}
\end{figure}
	
	Redibujando $\Gamma' \cup Q_{3}''$	(que es un grafo biconexo) y usando de nuevo el teorema \ref{teorema33} podemos asumir  que $Q_{3}''$ es un cuadrángulo con $Q_{1}'$ en su interior. Si ahora además elegimos $Q_{3}''$ como el nuevo $Q_{1}(p_k)$, entonces cualquiera de los $Q_{1}(p_1),..., Q_{1}(p_k)$ tiene intersecciones finitas dos a dos por ser finito el número de segmentos muy malos. Hemos cerrado la inducción.
 
	En consecuencia podemos asumir que que hay un número finito de segmentos muy malos dentro de cada $Q_{2}(p_k)$ y que además son arcos poligonales simples formando un grafo plano biconexo con $Q_{2}(p_k)$.

	La unión $\Delta:=\bigcup_{i = 1}^{n} Q_{1}(p_i)$ se puede ver como un grafo  en $S$ y cada región de $S \backslash \Delta$ está rodeada por un ciclo $C$ de $\Delta$. Ahora dibujamos un polígono convexo $C'$ de longitud 1 donde las esquinas de éste coincidan con los vértices de $\Delta$ en  $C$. Estos polígonos $C'$ se pueden suponer disjuntos dos a dos y conforman un subespacio $X'$ del plano. Identificando a pares los puntos de los lados de los polígonos en $X'$ con el criterio de {\em provenir del mismo punto} en una  arista de $\Delta$, el espacio identificación  $S'$ así obtenido es una superficie topológica, y el  grafo $\Delta'$ embebido bicelular asociado a esta identificación es naturalmente isomorfo a $\Delta$.

	Un isomorfismo de $\Delta$ a $\Delta'$ puede ser extendido a un homeomorfismo $f$ del conjunto de puntos de $\Delta$ en $S$ al conjunto de puntos de $\Delta'$ en $S'$. El teorema de Jordan-Schönflies nos permite extender $f$   a un homeomorfismo de $\overline{Int}(C)$ a $\overline{Int}(C')$ para cualesquiera celdas $C$ y $C'$, luego a un homeomorfismo de $S$ en $S'$. Esto prueba que $S$ es homeomorfa a una superficie triangulable, y concluye el teorema.
\end{proof}

	$^{1}$ La existencia de $Q_{3}''$ se puede establecer del siguiente modo: Para cada punto $p \in Q_{3}'$ definimos $R(p)$ un cuadrado que tiene a $p$ como punto medio y que además no corta los segmentos malos, excepto los que son muy malos. Consideramos entonces un recubrimiento finito de $Q_{3}'$ por cuadrados. La unión de estos cuadrados es un grafo plano biconexo cuyo ciclo exterior juega el papel de $Q_{3}''$.


\begin{remark} Como consecuencia directa del Teorema de Radó, toda superficie topológica compacta es homeomorfa a un polígono con lados identificados a pares. A partir de este hecho básico, y mediante métodos de cirugía topológica elemental, es posible demostrar que una superficie topológica compacta distinta de la esfera es homeomorfa a una suma conexa de toros o a una suma conexa de projectivos, siendo la unión de estas familias normalizadas de superficies exahustiva (esto es, no hay dos de ellas homeomorfas entre sí). Como corolario, toda superficie compacta se determina unívocamente por su característica de Euler y su carácter de orientabilidad; ver por ejemplo \cite{Massey}. Estos hechos forman parte del material abarcado por la asignatura Topología II del actual Grado en Matemáticas, por lo que no hemos considerado necesario redundar sobre ellos en este Trabajo Fin de Grado.
\end{remark}


\bibliographystyle{unsrt}
\begin{thebibliography}{10}

	\bibitem{Herrero} \textsc{P. J. Herrero Piñeyro},
	\textit{"Topología de Espacios Métricos"}, págs. 9-18, Universidad de Murcia, 2010.
	\bibitem{Massey} \textsc{W. S. Massey},
	\textit{"Algebraic Topology: An Introduction"}, Springer-Verlag, 1977.
	\bibitem{Stadler} \textsc{M. M. Stadler},
	\textit{"¿Qué es la Topología?"}, Universidad del País Vasco-Euskal Herriko Unibertsitatea, 2002.
	\bibitem{Kuratowski} \textsc{C. Thomassen},
	\textit{"Kuratowski's Theorem"}, J. Graph Theory, págs. 225-241, 1981.
	\bibitem{Thomassen} \textsc{C. Thomassen},
	\textit{"The Jordan-Schönflies Theorem and the Classification of Surfaces"}, American Mathematical Monthly, págs. 116-130, 1992.
	\bibitem{Tutte} \textsc{W. T. Tutte},
	\textit{"Graph Theory"}, Addison Wesley, 1984.
	
\end{thebibliography}

\end{document}
